\begin{definition} [Spatial domain] \hfill
    The section of the real plane spanned by the coordinates of an image is
    called the spatial domain.
\end{definition}

\begin{definition} [Spatial coordinates] \hfill
    $x$ and $y$ coordinates of spatial domain (also are called as spatial
    variables).
\end{definition}

\section{Intensity Transformations vs Spatial Filtering}

\paragraph*{} Image processing \textbf{transform}s data (intensity) of pixels to
another data.

\paragraph*{}Both \emph{spatial processing} and \emph{transform processing}
transformations do transformatinos. So what is the different? 

\paragraph*{} Incorrect thought about spatial processing ways:
\begin{enumerate}
    \item \st{work on pixels and we know intensity of each pixel will be
    transformed to what}
    \item \st{Directly change the intensity of pixels by a transformation.}
\end{enumerate}

\paragraph*{}{Specification of \emph{transform processing} ways: \newline
First transforming an image into the transform domain, doing the processing
there, and obtaining the inverse transform to bring the results back into the
spatial domain.}

\paragraph*{} In both \emph{Spatial processing} and \emph{transform processing}
transformations, transformation may be a combination of multiple
transformations; but in spatial processing the \textbf{meaning of pixel}s
remains. It means after any transformation, the result is an image, and so the
result of all transformations. But in \emph{transform domain}, there is a time
you do a transformation but its result is not an image (when you transform
image to transform domain); although the result of all transformations
in a \emph{transform domain} is also an image.

\section{Spatial processing ways}

\begin{itemize}
\item
  Intensity transformations\\
  (point processing)\\
  Its inputs are all pixels with same intensity\\
  In intensity transformations, $s=T(r)$ shows the output only depends on
  intensity of pixel $\Leftrightarrow$  all pixels with same intensity will have
  same output.\\
  \emph{usage}:contrast manipulation and image thresholding
\item
  Spatial filtering\\
  (neighborhood processing)\\
  Performs operations on the neighborhood of every pixel\\
  \emph{usage}:image smoothing and sharpening
\end{itemize}

\newglossaryentry{spatial_domain_glossary_label:intensity compression}
{
type={spatial_domain_glossary_label},
name={intensity compression},
description={image is washed-out more darker}}

\newglossaryentry{spatial_domain_glossary_label:intensity widening}
{
type={spatial_domain_glossary_label},
name={intensity widening},
description={(intensity expanding) more lighter image}}

% \gls*{intensity compression}
% \gls*{intensity widening}

% Use the terms
include \gls{spatial_domain_glossary_label:intensity compression} and \gls{spatial_domain_glossary_label:intensity widening}.
 
%\begingroup\let\newpage\relax
\printglossary[type=spatial_domain_glossary_label]
%\endgroup