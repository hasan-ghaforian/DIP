\subsection{What is brightness histogram?}

\paragraph*{} The \textbf{brightness histogram} $h_f(z)$ of an image $f$
provides the frequency of the intensity value $z$ in the image. The histogram of
an image with $L$ gray-levels is represented by a one-dimensional array with $L$
elements. If image $f$ be distinct, we may use $h(z)$.

\paragraph*{} Let $r_k$ , for $k = 0, 1, 2, … , L - 1$, denote the intensities
of an $L$-level digital image $f(x,y)$. Then \emph{unnormalized histogram} of
$f$ is defined as: 

\[
    h_f(r_k) = n_k \text{ when } n_k \text{ is the number of pixels in } f \text{ with intensity } r_k
\]
We call $r_k$s as \emph{histogram bin}s of $f$.

If size of $f$ is $MN$ then its \emph{normalized} histogram is defined as:
\[
    p_f(r_k) = \frac{h_f(r_k)}{MN}
\]

The normalized histogram provides a natural bridge between images and a
probabilistic description. 

The histogram is usually the \emph{only global information} about the image
which is available. For example it is used when:
\begin{itemize}
\item gray-scale transformations
\item image segmentation to objects and background    
\end{itemize}

\subsubsection*{Same histogram for different images}

Histogram show the frequency of intensities in each image. So if two images
have the same frequences for the same histogram bins, then their histograms will
be the same. This may occur in different situations; but I think all of them are
relative to \emph{permutation of pixels} in the $MN$ spatial domain. For
example: 
\begin{itemize}
    \item the histogram of $[0 \ 1]$ binary image is the same as $[1 \ 0]$
    image, although images are different. 
    \item change
    of the object position on a constant background does not affect the
    histogram of image.
\end{itemize}


\subsection{Entropy}

\[
    H(X) = -\sum\limits_{i=1}^{L-1}p(x_i)\log_2(p(x_i))
\]

\subsection{Affect of point processing on histograms}

As previously we saw in \hyperref[intensity transformation definition]{intensity
transformation definition}, all pixels with same intensity will have same output
under a point processing transformation. But $1-1$ property of point processing
transformation is not guaranteed. So it is possible that pixels of $r_1$ and
$r_2$ bins map to the same $s_0$ bin.  In this case, the height of histogram at
$s_0$ will be the sum of heights of histogram of source image at all $r_i$s
which are mapped to $s_0$. But if the transformation $T$ be $1-1$ and transform
image $f$ to the $g$; for example
when $T$ is monotonically increasing, then:
\[
    h_f(r_k) = h_g(T(r_k))
\]
This means the height of a bin and its image under the $T$ will be the same. See
\autoref{fig:1-1 point transformation keeps height of bars}. So in this cases,
\textbf{only} the position of bins can be changed and hence shape of histogram
may be \emph{shrinkage} or \emph{stretch} horizontally in different places
(\st{vertically}). Later we will see examples of shrinkage (compressing) and
stretch (expanding) of histograms.

But not all $1-1$ point processing transformations shrinkage or stretch the
histogram. For example $T = r + c$, when is a $1-1$ transformation\footnote{If
$r+c$ exceeds bounds of $[0,L-1]$, then it may be not $1-1$.}, only moves the
histogram horizontally. 


\begin{figure}[htb!]
    \begin{tikzpicture}[x=0.75pt,y=0.75pt,yscale=-0.5,xscale=0.5]
        %uncomment if require: \path (0,300); %set diagram left start at 0, and has height of 300
            
            %Shape: Axis 2D [id:dp9231531780387925] 
            \draw  (50,280.35) -- (345,280.35)(79.5,54) -- (79.5,305.5) (338,275.35) -- (345,280.35) -- (338,285.35) (74.5,61) -- (79.5,54) -- (84.5,61)  ;
            %Curve Lines [id:da9193569106111694] 
            \draw    (79.5,280.35) .. controls (90,103.5) and (275,22.5) .. (302,14.5) ;
            %Straight Lines [id:da905539517417276] 
            \draw [color={rgb, 255:red, 188; green, 27; blue, 27 }  ,draw opacity=1 ] [dash pattern={on 4.5pt off 4.5pt}]  (180.43,88.43) -- (181.15,280.15) ;
            %Straight Lines [id:da275519737489504] 
            \draw [color={rgb, 255:red, 45; green, 53; blue, 107 }  ,draw opacity=1 ][fill={rgb, 255:red, 82; green, 134; blue, 197 }  ,fill opacity=1 ]   (180.75,196.38) -- (181.15,280.15) ;
            %Straight Lines [id:da8637126099678687] 
            \draw  [dash pattern={on 0.84pt off 2.51pt}]  (79.57,88.71) -- (180.43,88.43) ;
            %Straight Lines [id:da3425284278838454] 
            \draw  [dash pattern={on 0.84pt off 2.51pt}]  (79.57,196.71) -- (180.43,196.43) ;
            %Shape: Axis 2D [id:dp6424593616756209] 
            \draw  (562,280.35) -- (857,280.35)(591.5,54) -- (591.5,305.5) (850,275.35) -- (857,280.35) -- (850,285.35) (586.5,61) -- (591.5,54) -- (596.5,61)  ;
            %Straight Lines [id:da2616151731156716] 
            \draw [color={rgb, 255:red, 45; green, 53; blue, 107 }  ,draw opacity=1 ][fill={rgb, 255:red, 82; green, 134; blue, 197 }  ,fill opacity=1 ]   (784.74,196.57) -- (785.14,280.35) ;
            %Straight Lines [id:da6457821008770631] 
            \draw  [dash pattern={on 0.84pt off 2.51pt}]  (591.57,196.71) -- (784.74,196.57) ;
            %Shape: Circle [id:dp04840312389851564] 
            \draw   (397.86,280.35) .. controls (397.86,173.4) and (484.55,86.71) .. (591.5,86.71) .. controls (698.45,86.71) and (785.14,173.4) .. (785.14,280.35) .. controls (785.14,387.3) and (698.45,473.99) .. (591.5,473.99) .. controls (484.55,473.99) and (397.86,387.3) .. (397.86,280.35) -- cycle ;
            %Straight Lines [id:da44634990042210343] 
            \draw  [dash pattern={on 3.75pt off 3pt on 7.5pt off 1.5pt}]  (79.5,280.35) -- (200,160.13) ;
            
            % Text Node
            \draw (174,282.4) node [anchor=north west][inner sep=0.75pt]    {$r_{1}$};
            % Text Node
            \draw (54,80.4) node [anchor=north west][inner sep=0.75pt]    {$s_{1}$};
            % Text Node
            \draw (50.43,187.54) node [anchor=north west][inner sep=0.75pt]    {$n_{r_{1}}$};
            % Text Node
            \draw (686,282.4) node [anchor=north west][inner sep=0.75pt]    {$r_{1}$};
            % Text Node
            \draw (566,80.4) node [anchor=north west][inner sep=0.75pt]    {$s_{1}$};
            % Text Node
            \draw (562.43,187.54) node [anchor=north west][inner sep=0.75pt]    {$n_{s_{1}}$};
            % Text Node
            \draw (779,283.4) node [anchor=north west][inner sep=0.75pt]    {$s_{1}$};
            
            
    \end{tikzpicture}
    \centering
    \caption{\emph{1-1} point transformation keeps height of bars}
    \label{fig:1-1 point transformation keeps height of bars}
\end{figure}

\clearpage